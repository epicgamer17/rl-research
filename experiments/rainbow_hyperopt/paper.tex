\documentclass{article}

% if you need to pass options to natbib, use, e.g.:
%     \PassOptionsToPackage{numbers, compress}{natbib}
% before loading arxiv

% ready for submission
% \usepackage{arxiv}

% to compile a preprint version, e.g., for submission to arXiv, add add the
% [preprint] option:
% \usepackage[preprint]{arxiv}

% to compile a camera-ready version, add the [final] option, e.g.:
\usepackage[final]{arxiv}

% to avoid loading the natbib package, add option nonatbib:
% \usepackage[nonatbib]{arxiv}

\usepackage[utf8]{inputenc} % allow utf-8 input
\usepackage[T1]{fontenc}    % use 8-bit T1 fonts
\usepackage{hyperref}       % hyperlinks
\usepackage{url}            % simple URL typesetting
\usepackage{booktabs}       % professional-quality tables
\usepackage{amsfonts}       % blackboard math symbols
\usepackage{nicefrac}       % compact symbols for 1/2, etc.
\usepackage{microtype}      % microtypography
\usepackage{xcolor}         % colors
\usepackage{cleveref}       % smart cross-referencing
\usepackage{lipsum}         % Can be removed after putting your text content
\usepackage{graphicx}
\usepackage{doi}

\title{Rainbow DQN Hyperparameter Optimization using Hyperopt for CartPole-v1
}

% Here you can change the date presented in the paper title
%\date{September 9, 1985}
% Or remove it
%\date{}

\newif\ifuniqueAffiliation
% Comment to use multiple affiliations variant of author block
\uniqueAffiliationtrue

% The \author macro works with any number of authors. There are two commands
% used to separate the names and addresses of multiple authors: \And and \AND.
%
% Using \And between authors leaves it to LaTeX to determine where to break the
% lines. Using \AND forces a line break at that point. So, if LaTeX puts 3 of 4
% authors names on the first line, and the last on the second line, try using
% \AND instead of \And before the third author name.

\ifuniqueAffiliation % Standard variant of author block
\author{%
  Jonathan Lamontagne-Kratz\\
  McGill University\\
  \texttt{jonathan.lamontagnekratz@mail.mcgill.ca} \\
  \And
  {Ezra Huang} \\
  McGill University\\
  \texttt{ezra.huang@mail.mcgill.ca} \\
}
\else
% Multiple affiliations variant of author block
\usepackage{authblk}
\renewcommand\Authfont{\bfseries}
\setlength{\affilsep}{0em}
% box is needed for correct spacing with authblk
\author[1]{%
	{\usebox\hspace{1mm}Jonathan Lamontagne-Kratz{\texttt{jonathan.lamontagnekratz@mail.mcgill.ca}}}%
}
\author[1,2]{%
    {\usebox\hspace{1mm}Ezra Huang{\texttt{ezra.huang@mail.mcgill.ca}}}%
}
\fi

% Uncomment to override  the `A preprint' in the header
\renewcommand{\headeright}{Technical Report}
\renewcommand{\undertitle}{Technical Report}
\renewcommand{\shorttitle}{Rainbow DQN HyperOpt}

%%% Add PDF metadata to help others organize their library
%%% Once the PDF is generated, you can check the metadata with
%%% $ pdfinfo template.pdf
\hypersetup{
pdftitle={Rainbow DQN Hyperparameter Optimization using Hyperopt for CartPole-v1},
pdfsubject={},
pdfauthor={Jonathan Lamontagne-Kratz, Ezra Huang},
pdfkeywords={Hyperopt, Rainbow DQN, Hyperparameter Optimization, CartPole-v1},
}

\begin{document}

\maketitle

\begin{abstract}
    In this paper, we present a hyperparameter optimization study of the Rainbow DQN algorithm using the Hyperopt library. We use the CartPole-v1 environment from the OpenAI Gym as a benchmark. We show that Hyperopt can be used to find optimal hyperparameters for the Rainbow DQN algorithm. We also confirm that the Rainbow DQN algorithm can be used to solve the CartPole-v1 environment with a high degree of success. We provide a detailed analysis of the hyperparameters found by Hyperopt and discuss the implications of these results. We explore how the definition of the search space can affect the convergence of hyperparameter optimization and quality of parameters. Additionaly, we present ways of comparing different parameter trials. We explore whether providing sucessful parameters from other games will decrease convergence times of a hyperparameter search, and compare different evaluation methods for hyperparameter optimization. Finally, we discuss the limitations of our study and suggest directions for future research.
\end{abstract}

% keywords can be removed
% \keywords{
%     Hyperopt \and
%     Rainbow DQN \and
%     Hyperparameter Optimization \and
%     CartPole-v1
% }


\section{Introduction}
\lipsum[2]
\lipsum[3]


\section{Background}
\label{sec:headings}

\lipsum[4] See Section \ref{sec:headings}.

\subsection{Deep Q-Neworks}
\lipsum[5]
\begin{equation}
  \xi _{ij}(t)=P(x_{t}=i,x_{t+1}=j|y,v,w;\theta)= {\frac {\alpha _{i}(t)a^{w_t}_{ij}\beta _{j}(t+1)b^{v_{t+1}}_{j}(y_{t+1})}{\sum _{i=1}^{N} \sum _{j=1}^{N} \alpha _{i}(t)a^{w_t}_{ij}\beta _{j}(t+1)b^{v_{t+1}}_{j}(y_{t+1})}}
\end{equation}

\subsection{Double DQN}
\lipsum[6]

\subsection{Prioritized Experience Replay}
\lipsum[7]

\subsection{Dueling DQN}
\lipsum[8]

\subsection{Noisy Nets}
\lipsum[9]

\subsection{Categorical DQN}
\lipsum[10]

\subsection{Rainbow DQN}
\lipsum[11]

\section{Methodology}
We used hyperopt with PTE (Parzen Tree Estimator) to optimize the hyperparameters of the Rainbow DQN algorithm. We used the CartPole-v1 environment from the OpenAI Gym as a benchmark. Each agent was trained for 10,000 training steps. Trials took about 10 minutes to complete on a single CPU and GPU. We used hyperparameter search spaces of different sizes in order to determin the effect of search space size on convergence time and performance. The different search spaces can be seen in the appendix. As a baseline we trained an agent using the hyperparameters seen in Table \ref{tab:table}. We also compared the performance of the Rainbow DQN algorithm with the performance of other previous DQN algorithms on the CartPole-v1 environment, as well as removed individual components of the Rainbow DQN algorithm to determine their effect on performance, as in \citep{DBLP:journals/corr/abs-1710-02298}.

\begin{table}
  \caption{Baseline hyperparameters}
  \label{tab:table}
  \centering
  \begin{tabular}{lll}
    \toprule
    \cmidrule(r){1-2}
    Name     & Value           \\
    \midrule
    Optimizer & Adam \\
    Adam $\epsilon$ & 1e-08  \\
    Learning rate & 0.001  \\
    Clipnorm & None  \\
    Loss function & Categorical crossentropy  \\
    Activation & ReLU  \\
    Kernel initializer & Orthogonal  \\
    Dense layers widths & 128  \\
    Replay interval & 1  \\
    Discount factor & 0.99  \\
    Minibatch size & 128  \\
    Replay buffer size & 5000  \\
    Min replay buffer size & 128  \\

    Target update interval & 100  \\

    Prioritized experience replay $\alpha$ & 0.2  \\
    Prioritized experience replay $\beta$ & 0.6  \\
    Prioritized experience replay $\epsilon$ & 1e-06  \\

    Value hidden layers widths & 128  \\
    Advantage hidden layers widths & 0  \\

    Noisy $\sigma$ & 0.5  \\

    N-step & 3  \\

    Atom size & 51  \\
    \bottomrule
  \end{tabular}
\end{table}


\section{Results}
\label{sec:others}

\lipsum[7]
See Figure~\ref{fig:fig1}.

\begin{figure}
  \centering
  \fbox{\rule[-.5cm]{0cm}{4cm} \rule[-.5cm]{4cm}{0cm}}
  \caption{Results of baseline agent.}
  \label{fig:fig1}
\end{figure}

\begin{figure}
  \centering
  \fbox{\rule[-.5cm]{0cm}{4cm} \rule[-.5cm]{4cm}{0cm}}
  \caption{Convergence of hyperopt on search space of size X, Y and Z.}
  \label{fig:fig2}
\end{figure}

\begin{figure}
  \centering
  \fbox{\rule[-.5cm]{0cm}{4cm} \rule[-.5cm]{4cm}{0cm}}
  \caption{Comparison of performance of Rainbow DQN with other DQN algorithms.}
  \label{fig:fig3}
\end{figure}

\begin{figure}
  \centering
  \fbox{\rule[-.5cm]{0cm}{4cm} \rule[-.5cm]{4cm}{0cm}}
  \caption{Effect of removing individual components of Rainbow DQN on performance.}
  \label{fig:fig4}
\end{figure}

\section{Discussion}


All artwork must be neat, clean, and legible. Lines should be dark enough for
purposes of reproduction. The figure number and caption always appear after the
figure. Place one line space before the figure caption and one line space after
the figure. The figure caption should be lower case (except for first word and
proper nouns); figures are numbered consecutively.

You may use color figures.  However, it is best for the figure captions and the
paper body to be legible if the paper is printed in either black/white or in
color.

\subsection{Lists}

\begin{itemize}
  \item Lorem ipsum dolor sit amet
  \item consectetur adipiscing elit.
  \item Aliquam dignissim blandit est, in dictum tortor gravida eget. In ac rutrum magna.
\end{itemize}

\subsection{Math}

Note that display math in bare TeX commands will not create correct line numbers for submission. Please use LaTeX (or AMSTeX) commands for unnumbered display math. (You really shouldn't be using \$\$ anyway; see \url{https://tex.stackexchange.com/questions/503/why-is-preferable-to} and \url{https://tex.stackexchange.com/questions/40492/what-are-the-differences-between-align-equation-and-displaymath} for more information.)

\subsection{Margins in \LaTeX{}}

Most of the margin problems come from figures positioned by hand using
\verb+\special+ or other commands. We suggest using the command
\verb+\includegraphics+ from the \verb+graphicx+ package. Always specify the
figure width as a multiple of the line width as in the example below:
\begin{verbatim}
   \usepackage[pdftex]{graphicx} ...
   \includegraphics[width=0.8\linewidth]{myfile.pdf}
\end{verbatim}
See Section 4.4 in the graphics bundle documentation
(\url{http://mirrors.ctan.org/macros/latex/required/graphics/grfguide.pdf})

A number of width problems arise when \LaTeX{} cannot properly hyphenate a
line. Please give LaTeX hyphenation hints using the \verb+\-+ command when
necessary.

% \begin{ack}
% Use unnumbered first level headings for the acknowledgments. All acknowledgments
% go at the end of the paper before the list of references.

% Do {\bf not} include this section in the anonymized submission, only in the final paper. You can use the \texttt{ack} environment provided in the style file to autmoatically hide this section in the anonymized submission.
% \end{ack}

\bibliographystyle{unsrtnat}
\bibliography{references}

%%% Uncomment this section and comment out the \bibliography{references} line above to use inline references.
% \begin{thebibliography}{1}
%
% \bibitem[\protect\citeauthoryear{Kour}{2014}]{kour2014real}
% George Kour and Raid Saabne.
% \newblock Real-time segmentation of on-line handwritten arabic script.
% \newblock In {\em Frontiers in Handwriting Recognition (ICFHR), 2014 14th
%   International Conference on}, pages 417--422. IEEE, 2014.
%
% \bibitem[\protect\citeauthoryear{Kour}{2014}]{kour2014fast}
% George Kour and Raid Saabne.
% \newblock Fast classification of handwritten on-line arabic characters.
% \newblock In {\em Soft Computing and Pattern Recognition (SoCPaR), 2014 6th
%   International Conference of}, pages 312--318. IEEE, 2014.
%
% \bibitem[\protect\citeauthoryear{Keshet}{2016}]{keshet2016prediction}
% Keshet, Renato, Alina Maor, and George Kour.
% \newblock Prediction-Based, Prioritized Market-Share Insight Extraction.
% \newblock In {\em Advanced Data Mining and Applications (ADMA), 2016 12th International
%   Conference of}, pages 81--94,2016.
%
% \end{thebibliography}

%%%%%%%%%%%%%%%%%%%%%%%%%%%%%%%%%%%%%%%%%%%%%%%%%%%%%%%%%%%%

\end{document}
